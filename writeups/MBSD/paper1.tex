\newcommand{\real}{\mathbb{R}}
\newcommand{\loc}{\mathit{Loc}}
\newcommand{\var}{\mathit{Var}}
\newcommand{\lab}{\mathit{Lab}}
\newcommand{\edg}{\mathit{Edg}}
\newcommand{\act}{\mathit{Act}}
\newcommand{\evol}{\mathit{Evol}}
\newcommand{\invr}{\mathit{Inv}}
\newcommand{\trans}{\mathit{(l,a,\mu,l')}}
\newcommand{\reach}{\mapsto^*}
\newcommand{\ftclosure}[2]{\langle #1 \rangle^{\nearrow}_{#2}}
\newcommand{\btclosure}[2]{\langle #1 \rangle^{\swarrow}_{#2}}
\newcommand{\lfp}{\mathit{lfp}}
\newcommand{\wide}{\nabla}
\newcommand{\pred}{\triangleright}

\section{Paper 1: The algorithmic analysis of hybrid systems \cite{Alur:95}} \label{sec:paper1}

The paper discusses semi-decision procedures for linear hybrid systems for performing reachability analysis and 
model checking of \emph{Timed Computation Tree Logic} (TCTL). 

\textbf{Linear Hybrid Systems:}
A hybrid system $H$ is a tuple $(\loc,\var,\lab,\edg,\act,\invr)$ where,
$\loc$ is a set of locations and $\var$ is a set of variables. 
Let $V= \var \mapsto \real$ denote a set of valuations which are functions mapping variables to real values.
$\lab$ is a set of labels (not significant for this discussion). $\edg$ is a set of transitions of the form $\trans$ with a source location $l \in \loc$, target location $l' \in \loc$, a label $a \in \lab$ and a transition relation $\mu \subseteq V \times V$ that captures the changes in the values of the variables during the transition. 
$\act \in \loc \mapsto \evol$ assigns each location to an evolution function $\evol = V \times \real^+ \mapsto V$ that describe the evolution of the variables with respect to the time
\footnote{The paper uses a more general definition for the evolution functions but since the focus is only on \emph{time deterministic systems} we use this simplified formalization in this report}. 
We use $\act_l$ to denote the activity of a location $l$.
Finally, each location $l$ is associated with an invariant given by $\invr(l) \subseteq V$ that characterizes the set of allowed valuations for the location $l$.

The states $\Sigma$ of a hybrid system is the set of location, valuation pairs i.e, $\Sigma = \loc \times V$.
Define a region $R$ as a subset of states.
We say that \emph{time can progress} til time $t$ from a valuation $v$ in a location $l$ denoted as 
$tcp_l(v,t)$ iff $\forall 0 \le t' \le t. \; \act_l(v,t') \in \invr(l)$.
A run of a hybrid system is a sequence of states $\sigma_0 \mapsto^{t_0} \sigma_1 \mapsto^{t_1} \sigma_2 \mapsto^{t_2} \cdots$ 
where, for all $i \ge 0$, $\sigma_i = (l_i,v_i) \in \Sigma$, $t_i \in \real^+$,
$tcp_{l_i}(v_i,t_i)$ and
$\sigma_{i+1}$ is the transition successor of $\sigma_i$.

The paper defines a linear hybrid system as a restricted hybrid system in which:
(a) for each location $l$, $\act_l(v,t) = v'$ where for all $x \in \var$, $v'(x) = v(x) + k_x t$. That is, each variable $x$ changes with a constant rate $k_x$ with respect to time.
(b) for each location $l$ the invariant $\invr(l)$ is a linear formula over $\var$.
(c) For each transition $e \in \edg$, the transition relation $\mu$ is a guarded set of nondeterministic assignments of the form $\psi \Rightarrow \{ x := [\alpha_x,\beta_x] \vbar x \in \var \}$ for some real constants $\alpha_x$ and $\beta_x$.

\begin{figure}

\caption{A power controller system of a processor.  
The locations correspond to low and high frequency modes.} \label{fig:hsys}
\end{figure}

As illustrated in the paper many interesting hybrid systems are linear.
Here, we discuss a new example shown in  Figure~\ref{fig:hsys} that models a power controller 
of a processor described below.
Consider a processor that can operate in two frequency modes, namely, the high and low frequency modes.
The power consumption $P(t)$ of the processor is a function of time that increases at a constant rate $P_{high}$ and
$P_{low}$ in the high and low frequency modes, respectively. 
Assume that the processor may switch from low to high frequency mode due to some external factors such as the load in the 
system. The power controller ensures that the total power consumed by the processor in 10 time units is atmost $P_{max}$. 

In Figure~\ref{fig:hsys}, $x_1, x_2$ are timers that track the time spent in low and high frequency modes, respectively, in an interval of 10 time units.
$p_1$ and $p_2$ record the total power consumed in the low and high frequency modes during a span of 10 time units.
We require that the safety property $p_1 + p_2 \le P_{max}$ always holds in the system .
Assume that $P_{low} \le P_{max}/10$ i.e, if the processor is run only in the low frequency mode then the property does hold.
The hybrid system shown in Figure~\ref{fig:hsys} is clearly linear.

\textbf{Reachability Analysis:}
The reachability analysis determines if a state $\sigma'$ is reachable from a state $\sigma$ i.e, if there exists
a run of H that starts in $\sigma$ and ends in $\sigma'$. If $R \subseteq \Sigma$ is unreachable from the initial state of the system then $\Sigma \setminus R$ is an invariant of the system.
Reachability problem of linear hybrid systems in undecidable. The paper proposes 3 semi-decision procedures for 
the reachability analysis which are discussed below.

\textbf{Forward and Backward analysis using Predicate Transformers:}
Define the forward time closure $\ftclosure{P}{l}$ of a location $l$ and a predicate $P \subseteq V$ as the set
$\{ v' \vbar \exists v \in V,t \in \real^+. v \in P \wedge tcp_l(v,t) \wedge v' = \act_l(v,t) \}$.
Similarly, define backwards time closure $\btclosure{P}{l}$ as the dual set 
$\{ v' \vbar \exists v \in V,t \in \real^+. v \in P \wedge tcp_l(v',t) \wedge v = \act_l(v',t) \}$.

Define the post operator of a transition $e = \trans$ as 
$post_e(P) = \{ v' \vbar \exists v \in V. v \in P \cap \invr(l) \wedge (v,v') \in \mu \wedge v' \in \invr(l') \}$.
The pre operator is defined in a dual way.

The set of reachable states from the initial states $I = (l,I_l)$ (where $l$ is the initial location
and $I_l \subseteq \invr(l)$) 
is defined as the least fix point ($\lfp$) of the set of equations 
$X_l = \ftclosure{I_l \cup \bigcup_{e=(l',a,\mu,l) \in \edg} post_e(X_{l'})}{l}$.
Similarly, we define the set of states that may lead to a final state $F = (l,F_l)$ (where $F_l \subseteq \invr(l)$) 
as the $\lfp$ of the set of equations $X_l = \btclosure{F_l \cup \bigcup_{e=(l,a,\mu,l') \in \edg} pre_e(X_{l'})}{l}$.

For the example shown in Figure~\ref{fig:hsys}, we can compute iteratively the set of reachable states from 
the initial state $(1,x_1=x_2=p_1=p_2=0)$ as illustrated below (we denote states using logical formulas):
%
\begin{align*}
X_{1,1} & = \ftclosure{x_1=x_2=p_1=p_2=0 \vee post_{(2,1)}(X_{2,0})}{l} \\
		& = \exists x_1,x_2,p_1,p_2. t \in \real^+. \; x_1=x_2=p_1=p_2=0   \\
		& \qquad \wedge x_1' = x_1 + t \wedge x_1 + t \le 10 \wedge p_1' = p_1 + t P_{low} \\ 
		& \qquad \wedge p_2' = p_2 \wedge x_2' = x_2\\
		& = 0 \le x_1' \le 10 \wedge x_2' = p_2' = 0 \wedge p_1' = x_1' P_{low} \\
X_{2,1} &= \ftclosure{false \vee post_{(1,2)}(X_{1,0})}{l} = false  \\
X_{2,2} & = 0 \le x_1' \le 10 \wedge 0 \le x_2' \le 10 - x_1' \wedge p_1' = x_1'P_{low}  \\
		& \qquad \wedge p_2' = x_2' P_{high} \wedge x_1'P_{low} + x_2'P_{high} \le P_{max} \\
X_{1,3} & = 0 \le x_1' \le 10 \wedge 0 \le x_2' \le 10 \wedge p_1' = x_1'P_{low} \wedge p_2' = 0 
\end{align*}
%
The fixpoint iteration converges at the third iteration with $X_{1,3}$ and $X_{2,3} = X_{2,2}$. 
It can be seen that the fixpoint $X = \bigvee_{l \in \loc} (pc = l \wedge X_l)$ implies the 
safety property $p_1 + p_2 \le P_{max}$ proving that it is an invariant (the implication holds
only under the assumption that $10P_{low} \le P_{max}$).

Similarly, it is possible to perform a backward analysis starting from the negation of the safety property
which would be $p_1 + p_2 > P_{max}$ and show that the intersection with the initial state is empty.
For this example, the backward fixpoint intersected with the initial state would yield $10 P_{low} > P_{max}$
(details elided for brevity) which by our assumption is not satisfiable.

\textbf{Approximate fixpoint computation using Abstract Interpretation:}
The fixpoint iteration described above may not always converge. Since the hybrid  systems are linear
the predicate transformers and time closures are convex polyhedra (after quantifier elimination).
The paper proposes to compute an approximation (a super set) of the fixpoint by approximating the 
union of polyhedra  using an abstract join operation $\bigsqcup$ which is defined as their \emph{convex hull}
(as discussed in \cite{pcousot:1975}).
To ensure convergence the paper proposes to use a widening operation $\wide$ at selected locations (that cut
all the loops in the automaton) in every iteration. For any two polyhedra (or linear constraints) 
$P,P'$, $P_1 \wide P_2$  is exactly those constraints of $P$ that are also satisfied by $P'$.

\textbf{Minimization}
This approach for reachability analysis computes the coarsest bisimulation of the transition system 
represented by the hybrid system with respect to a region $R_F$ (if one exists). The approach starts 
with the initial partition $(R_F,\Sigma - R_F)$ and checks if there exists a region that is not \emph{stable}
(a region is stable if all of its (time/transition) successor states belong to the same region). 
If not then $R$ is split into smaller states.

\textbf{Model checking}

The paper uses timed computation tree logic \emph{TCTL} to specify real time properties of hybrid systems.
Let $C$ be a set of clocks that are disjoint from $\var$. The TCTL formulas are given by the following grammar: 
$\phi ::= \psi \vbar \neg \phi \vbar \phi_1 \vee \phi_2 \vbar z.\phi \vbar \phi_1 \exists \phi_2 \vbar \phi_1 \forall \phi_2$
where, $\psi$ is a state predicate defined over $\var \cup C$, $z \in C$ is a clock variable,
$\phi_1 \exists \phi_2$ is true if there exists a future in which $\phi_1$ holds until $\phi_2$ 
and $\phi_1 \forall \phi_2$ is true if in all futures $\phi_1$ holds until $\phi_2$.

The semantics of $\exists$ and $\forall$ can be defined using the fixpoint equations in the standard way.
For the purpose of this report, we present the fixpoint characterization only for 
the \emph{possibly} operator $true \exists \phi$.
$true \exists \phi = \bigcup_{i \ge 0} R_i$ where,
$R_0 = R_{\phi}$ the set of states satisfying $\phi$;
for all $i \ge 0$, $R_{i+1} = R_i \cup (\pred R_i)$ where $\pred R_i$ is the
set of predecessor states of $R_i$ defined as the set 
$\{ (l,v) \vbar \exists (l',v') \in R_i, t \in \real^+. \; (l,v) \mapsto^t (l',v') \wedge tcp_l(v,t) \}$
\footnote{This definition is an adaptation of the definition for $\phi_1 \exists \phi_2$ given in the paper 
to the \emph{possibly} operation}.

For our example, we illustrate this approach for the formula $true \exists \;(p_1 + p_2 > P_{max})$ which 
states that exists a future in which the total power consumed is more than $P_{max}$.
Let $P_{max} = 10, P_{low} = 1, P_{high} = 2$.
We use formulas to represent states in the following.
%
\begin{align*}
R_0 &= (pc = 1 \wedge p_1 + p_2 > 10 \wedge x_1 \le 10) \vee \\
	& (pc = 2 \wedge p_1 + p_2 > 10 \wedge x_2 \le 10 - x_1 \wedge x_1+2x_2 \le 10 \\
R_1 &= R_0 \vee false \vee \exists x_1',x_2',p_1',p_2',t. \; (pc=1 \wedge t \ge 0 \wedge \\
	& x_2'=0 \wedge p_1'+p_2'>10 \wedge x_2' \le 10 - x_1' \wedge x_1'+2x_2' \le 10 \wedge \\
	& x_1' = x_1 + t \wedge p_1' = p_1 + t \wedge x_2' = x_2 \wedge p_2' = p_2 \\
	&= R_0 \vee (pc=1 \wedge p_1+p_2 >x_1 \wedge x_2 = 0) \\
R_2 &= R_1 \vee false	= R_1
\end{align*}
%
Thus, the fixpoint iteration terminates at iteration 3. The set of all states that satisfy the 
CTL formula is given by the state predicate $R_1$. Notice that the initial state  
$(pc=1 \wedge p_1=p_2=x_1=x_2=0)$ intersected with $R_1$ is false i.e, the CTL formula is never
satisfiable starting from the initial state, which proves that $(p_1 + p_2 \le P_{max})$ is an invariant.
Interestingly, other initial states that satisfy $R_1$ may lead to the violation of the invariant
e.g. $(pc=1 \wedge p_1=p_2=10 \wedge x_1=x_2=0)$.