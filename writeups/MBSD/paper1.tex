\newcommand{\real}{\mathbb{R}}
\newcommand{\loc}{\mathit{Loc}}
\newcommand{\var}{\mathit{Var}}
\newcommand{\lab}{\mathit{Lab}}
\newcommand{\edg}{\mathit{Edg}}
\newcommand{\act}{\mathit{Act}}
\newcommand{\evol}{\mathit{Evol}}
\newcommand{\invr}{\mathit{Inv}}
\newcommand{\trans}{\mathit{(l,a,\mu,l')}}
\newcommand{\reach}{\mapsto^*}

\section{Paper 1: The algorithmic analysis of hybrid systems \cite{Alur:95}} \label{sec:paper1}

The article proposes and discusses general approaches for the automated analysis of \emph{linear} hybrid systems by extending the standard techniques for program analysis and model checking of simple transition systems. The paper first discusses three semi-decision procedures for reachability analysis of hybrid systems which is in general undecidable. Subsequently, the paper discusses a (semi-decidable) approach  for model checking arbitrary formulas belonging to the \emph{Timed Computation Tree Logic} (TCTL) which is an extension of CTL.

\paragraph*{\textbf{Linear Hybrid Systems}} 

%Linear hybrid systems are a restriction of general hybrid systems in which the transition relations and \emph{evolution} functions (also called \emph{activities}) are linear.
%Formally, 
A hybrid system $H$ is a tuple $(\loc,\var,\lab,\edg,\act,\invr)$ where,
$\loc$ is a set of locations and $\var$ is a set of variables. 
Let $V= \var \mapsto \real$ denote a set of valuations which are functions mapping variables to real values.
$\lab$ is a set of labels (not very significant for this discussion). $\edg$ is a set of transitions of the form $\trans$ with a source location $l \in \loc$, target location $l' \in \loc$, a label $a \in \lab$ and a transition relation $\mu \subseteq V \times V$ that captures the changes in the values of the variables during the transition. 
$\act \in \loc \mapsto \evol$ assigns each location to an evolution function $\evol = V \times \real^+ \mapsto V$ that describe the evolution of the variables with respect to the time 
\footnote{The paper uses a more general definition for the evolution functions but since the focus is only on \emph{time deterministic systems} we use this simplified formalization in this report}. 
Finally, each location $l$ is associated with an invariant given by $\invr(l) \subseteq V$ that characterizes the set of allowed valuations for the location $l$.

The states $\Sigma$ of a hybrid system is the set of location, valuation pairs i.e, $\Sigma = \loc \times V$.
Define a region $R$ as a subset of states.
A run of a hybrid system is a sequence of states $\sigma_0 \mapsto^{t_0} \sigma_1 \mapsto^{t_1} \sigma_2 \mapsto^{t_2} \cdots$ 
where, for all $i \ge 0$, $\sigma_i = (l_i,v_i) \in \Sigma$, $t_i \in \real^+$,
$\forall 0 \le t \le t_i. \act(l_i)(v_i,t) \in \invr(l_i)$ and
$\sigma_{i+1}$ is the transition successor of $\sigma_i$.

The paper defines a linear hybrid system as a hybrid system in which:
(a) for each location $l$, $\act(l)(v,t) = v'$ where for all $x \in \var$, $v'(x) = v(x) + k_x t$. That is, each variable $x$ changes with a constant rate $k_x$ with respect to time.
(b) for each location $l$ the invariant $\invr(l)$ is a linear formula over $\var$.
(c) For each transition $e \in \edg$, the transition relation $\mu$ is a guarded set of nondeterministic assignments of the form $\psi \Rightarrow \{ x := [\alpha_x,\beta_x] \vbar x \in \var \}$ for some real constants $\alpha_x$ and $\beta_x$.

\begin{figure}

\caption{A power controller system of a processor.  
The locations correspond to low and high frequency modes.} \label{fig:hsys}
\end{figure}

As illustrated in the paper many interesting hybrid systems are linear. For instance, the \textit{water-level monitor systems, Fischer's mutual exclusion algorithm, temperate control systems} can be modeled as linear hybrid systems. 
We now discuss a different example shown in  Figure~\ref{fig:hsys} that models a power controller 
of a processor described below.
Consider a processor that can operate in two frequency modes, namely, the high and low frequency modes.
The power consumption $P(t)$ of the processor is a function of time that increases at a constant rate $P_{high}$ and
$P_{low}$ in the high and low frequency modes, respectively. 
Assume that the processor may switch from low to high frequency mode due to some external factors such as the load in the 
system. The power controller ensures that the total power consumed by the processor in 10 time units is atmost $P_{max}$. 

In Figure~\ref{fig:hsys}, $x_1, x_2$ are timers that track the time spent in low and high frequency modes, respectively, in an interval of 10 time units.
$p_1$ and $p_2$ record the total power consumed in the low and high frequency modes during a span of 10 time units.
We require that the safety property $p_1 + p_2 \le P_{max}$ always holds in the system .
Assume that $P_{low} \le P_{max}/10$ i.e, if the processor is run only in the low frequency mode then the property does hold.
The hybrid system shown in Figure~\ref{fig:hsys} is clearly linear.

\paragraph*{\textbf{Reachability Analysis}}



