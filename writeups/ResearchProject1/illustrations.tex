\section{Illustrations.}

Consider the procedure shown in Figure~\ref{fig:eg2} that now simultaneously computes the size and the 
length of the left most path of a binary tree.
%
\begin{figure}
\begin{myprogram}
s(t) \{ \\
\pnl \> t match \{ \\
\pnl \> case Emp() => (0,0); \\
\pnl \> case N(x,l,r) => \{ \\
\pnl \> \> (a,b) = s(l); (c,d) = s(r) \\
\pnl \> \> (a + c + 1,b+1); \\
	 \> \> \}
\> \} \\
\} ensuring(res => res.1 != res.2 - 1)
\end{myprogram}
\caption{An example program} \label{fig:eg2}
\end{figure}
%
Applying the \IndGen algorithm for this example will result  in a condition that has the atomic predicate 
$a+c+1 \ne b$. Assume that we already have a counter-example $t \mapsto Emp()$. 
Invoking the \SG algorithm on this predicate with the counter-example will result in a formula
like $a \ge b - \epsilon_1 \wedge c \ge -\epsilon_2$, where $\epsilon_1$ and $\epsilon_2$ are some 
fractions between $0$ and $1$.
This is because \SG will rewrite the predicate $a+c+1 \ne b$ as $t + c + 1 \ne 0 \wedge t = a - b$  for
some fresh variable $t$ as $a$ and $b$ both belong to the same recursive call $(a,b) = s(l)$.
The \hypervol algorithm when invoked on $t + c + 0.9 \le 0$ (along with evaluations of the counter-example
$t \mapsto Emp()$) will generate $t \ge -\epsilon_1 \wedge c \ge -\epsilon_2$. 
Finally, replacing $t$ by $a -b$ will result in $a \ge b - \epsilon_1 \wedge c \ge -\epsilon_2$.
This formula gets translated to the required inductive invariant 
$res.1 \ge res.2 - \epsilon_2 \wedge res.1 \ge -\epsilon_1$.
 
