\section{Research Proposal}

Automated verification of functional correctness properties is hard but there
exists great scope for extending the current state-of-the-art.
which is exemplified by the techniques discussed above whose behavior on the 
simple list reversal program is not very satisfactory.
We plan to start attacking this problem by extending the template based invariant generation 
techniques  proposed for numerical programs such as the one discussed in the 
section~\ref{sec:paper1} and also in \cite{ssriram:SAS04,ssriram:CAV03}. 

Our initial (short-term) objective is to support inference of invariant that belong to a predefined
template which may involve polynomials, user-defined (recursive) functions 
and data structures. For example, we would like to support templates of the form 
$a.f_1(t_1) + b.f_2(t_2) + c.f_3(t_3) = 0$ where, $a$,$b$ and $c$ are unknown coefficients (integers or reals),
$f_1$,$f_2$ and $f_3$ are unknown functions and $t_1$,$t_2$ and $t_3$ are unknown data types.
The inference engine has to bind the unknown coefficients to suitable values so that the verification 
succeeds. 
As in the previous works, we plan to solve the extended templates by efficiently encoding them as
solutions of logical constraints and solve them using dedicated constraint solvers 
(such as SMT solvers, BMI, LMI solvers etc.)
We plan to investigate constraint solving based techniques instead of iterative procedures such
as predicate abstraction for the following reasons 
(a) to exploit the ever growing advances in the constraint solving technology.
(b) to increase the applicability of the technique to problems beyond invariant inference. 
As discussed in section~\ref{sec:paper1} termination problem can also be modeled as constraint solving. 
Interestingly, it is also possible to cast \emph{synthesis problem} as constraint solving.
Developing efficient techniques for solving a set of constraints involving sophisticated
templates such as the one described above enables us to simultaneously target several interesting
related problems.
However, there several challenges to constraint solving techniques the most important of
which is the exponential blow-up in the sizes of the constraints to be solved when 
functions and data-structures are involved.
For example, \cite{dirk:VMCAI07} discusses a technique for handling linear templates extended with
just uninterpreted functions. Unfortunately, the sizes of the constraints are in the order of $n!$
(where $n$ is the size of the verification condition). 
Furthermore, the use of disjunctions in the program code or specification further increases 
the sizes of the constraints.
A huge practical challenge is to control the sizes of the constraints.
We plan to explore abstraction  and incrementalization (for instance, see \cite{}) to address this problem.

We plan to implement and integrate our approach with the \emph{Leon verification framework} 
\cite{psuter:SAS11}.
\emph{Leon} is verifier for proving specifications of functional programs involving 
recursive functions, arithmetics and data structures. 
Leon already incorporates a powerful interprocedural verification technique for handling 
programs with multiple procedures.
However, in its current form it cannot automatically  strengthen the specifications (by inferring inductive
invariants) if necessary. We plan to integrate our approach with Leon so that it can handle such scenarios. 