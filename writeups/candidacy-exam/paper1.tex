\newcommand{\real}{\mathbb{R}}

\section{Survey of the Selected Papers}

\subsection{Program invariance and termination using Lagrangian Relaxation and Semidefinite Programming}

\cite{cousot:VMCAI05} proposes an approach for inferring polynomial invariants that 
are of the form given by a user-defined \emph{template} which is a polynomial with unknown coefficients.
The technique is proposed for semi-algebraic programs which are numerical programs with real valued variables 
and iterative constructs of the form \texttt{while B do C} where B is a boolean condition 
and C is an imperative command. Figure~\ref{fig:SAprogram} shows a semialgebraic program that computes the least factorial greater than or equal to a given N.
%
\begin{figure}
\begin{myprogram}
\\
\pnl \>  n = 0 \\
\pnl \> f = 1 \\
\pnl \> while(f <= N) \{ \\
\pnl \> \> n = n+1 \\
\pnl \> \> f = n* f \\
\pnl \> \}
\end{myprogram}
\caption{A program computing the least factorial greater than or equal to a given N.} \label{fig:SAprogram}
\end{figure}
%
Assume that the (relational) semantics of an iteration is given by a formula of the form
$\bigwedge \limits_{k=1}^{n} (\sigma_k(x,x') \ge 0)$ where, $x$ and $x'$ are the vectors of
variables representing the values of the program variables before and after an iteration, respectively 
and $\sigma_k : \real^n \times \real^n \mapsto \real$ is a polynomial over $x$ and $x'$. 
In other words, it is assumed that the relational semantics of each
iteration can be expressed as a conjunction of polynomial positivity constraints.
For the program shown in Figure~\ref{fig:SAprogram} the relational semantics of the 
loop is given by $N-f \ge 0 \wedge n' -n -1 = 0 \wedge f' - n'.f = 0$ (where 
$A = 0$ is a short hand for $A \ge 0 \wedge A \le 0$).

An inductive invariant of a loop is another polynomial inequality $I(x) \ge 0$ 
that satisfies the following inductiveness constraints:
%
\begin{align}
& \forall x. P(x) \ge 0 \Rightarrow I(x) \ge 0 \\
& \forall x,x'. I(x) \ge 0 \wedge \bigwedge_k (\sigma_k(x,x') \ge 0) \Rightarrow  I(x') \ge 0
\end{align}
%
where $P(x) \ge 0$ is the precondition.
To prove that a property $S(x) \ge 0$ is an invariant of the loop we need to find an $I(x)$ such that
$\forall x'. I(x') \ge 0 \Rightarrow S(x') \ge 0$.
Assume that we are interested only in solutions of $I(x)$ having the form $I_a(x)$ which is a polynomial with 
unknown coefficients given by a vector $a$. 
For example, $I_a(x) = a.(x \; 1)^T$ represents all linear (or affine) invariants, 
$I_a(x) = (x \; 1).a.(x \; 1)^T$ represents all quadratic invariants  and so on. 
Using this assumption we can rewrite the above constraints as follows:
%
\begin{align}
& \nonumber \exists a \in \real^p:  \\
& \forall x. P(x) \ge 0 \Rightarrow I_a(x) \ge 0  \label{eq:template1}\\
& \forall x,x'. I_a(x) \ge 0  \wedge \bigwedge_k (\sigma_k(x,x') \ge 0) \Rightarrow  I_a(x') \ge 0 \label{eq:template2}
\end{align}
%
Hence,  the problem essentially reduces to finding a vector $a$ that satisfies
the above constraints.

The paper proposes to use \emph{Langrangian Relaxation} method to solve for $a$ in the above constraints. The method states that to prove 
$\forall v. \bigwedge \limits_{k=1}^{n} \sigma_k(v) \ge 0 \Rightarrow \sigma_0(v) \ge 0$ it suffices to prove that 
%
\begin{align}
\exists (\lambda_1 \cdots \lambda_n) \in (\real^+)^n. \; \forall v. \; \left( \sigma_0(v) -
\sum \limits_{k=1}^{n} \lambda_k \sigma_k(v) \right) \ge 0 \label{eq:lagrange}
\end{align}
%
By Lagrangian relaxation, the constraints~\ref{eq:template1} and \ref{eq:template2}
can be reduced to the following:
%
\begin{align}
& \nonumber \exists a \in \real^p, \exists \mu \in \real^+, \exists (\lambda_0 \cdots \lambda_n) \in (\real^+)^{n+1}  : \\
&\forall x. I_a(x) - \mu.P(x)  \ge 0  \label{eq:inv1}\\
&\forall x,x'. \left( I_a(x') - \lambda_0.I_a(x) - \sum \limits_{k=1}^{n} \lambda_k. \sigma_k(x,x') \right)  \ge 0 \label{eq:inv2}
\end{align}
%
Lagrangian relaxation is in general incomplete i.e, it is not a necessary condition.
However, when the relational semantics is linear i.e, for all $k$, 
$\sigma_k(v)$ is linear then it is complete by \emph{Farkas' lemma}.
In fact, as discussed in some the other related works \cite{ssriram:CAV03,ssriram:SAS04}, for the linear case the constraint~\ref{eq:lagrange} 
can be further reduced to a set of quadratic constraints on the 
coefficients of $\sigma_k$ using \emph{Farkas' Lemma}. 
By solving for the unknown coefficients (obtained due to the template $I_a$) in the
quadratic constraints one can infer an inductive invariant.
However, this is valid only for the linear case. We will shortly discuss
the solution proposed in the paper for solving constraints~\ref{eq:inv1} and \ref{eq:inv2}.

The paper also considers a related problem of proving termination of a program
and shows that it can also be reduced to finding a solution to constraints 
similar to \ref{eq:inv1} and \ref{eq:inv2}. A standard way of proving that 
a loop always terminates is to discover a function (called \emph{ranking function}) whose value strictly decreases at every loop iteration and remains positive across all iterations.
Furthermore, the decrease of the ranking function has to be bounded from below to avoid zeno phenomenon.
Formally, a function $r: \real^p \mapsto \real$ is a ranking function 
iff it satisfies the following constraints:
%
\begin{align*}
& \exists \delta \in \real:  \qquad \delta > 0 \\
& \forall x. \; I(x) \ge 0 \Rightarrow r(x) \ge 0 \\
& \forall x,x'. I(x) \ge 0  \wedge \bigwedge_k \sigma_k(x,x') \ge 0 \Rightarrow  
(r(x) - r(x') - \delta) \ge 0
\end{align*}
%
where $I(x)$ is a loop invariant which is assumed to be already inferred 
or given by the user. 
In plain words, the constraints enforce that $r(x) \ge 0$ is a loop invariant 
(note that it is implied by a known invariant) and that it strictly decreases by atleast $\delta$ in each loop iteration.

As in the case of invariants assume that the ranking function is also given by
a polynomial template $r_a(x)$ with unknown coefficients $a$. By the lagrangian
relaxation the ranking function constraints can be reduced to the following form:
%
\begin{align}
& \nonumber \exists \delta \in R, \exists a \in \real^p, \exists \mu \in \real^+, \exists (\lambda_0 \cdots \lambda_n) \in (\real^+)^{n+1} : \\
& \delta > 0 \label{eq:rank1}\\
& \forall x. \; r_a(x) - \mu.I(x)  \ge 0 \label{eq:rank2}\\
& \forall x,x'. \left( r_a(x) - r_a(x') - \delta - \lambda_0.I_a(x) - \sum \limits_{k=1}^{n} \lambda_k. \sigma_k(x,x') \right) \ge 0 \label{eq:rank3}
\end{align}
%


